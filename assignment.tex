% You should not modify anything from here ... -------------
\documentclass[a4paper]{article}
\usepackage[english]{babel}
\usepackage{microtype,etex,listings,color,parskip}
\usepackage[margin=2cm]{geometry}
\usepackage{hyperref}
\usepackage{amsmath, amssymb}
\usepackage{graphicx}
\usepackage{tikz}
\usepackage{optidef}
\lstset{
  language=C,
  tabsize=2,
  showstringspaces=false,
  breaklines=true,
  basicstyle=\ttfamily,
  keywordstyle=\color[rgb]{0.1,0.3,0.7}\ttfamily,
  stringstyle=\color[rgb]{0.7,0.1,0.3}\ttfamily,
  commentstyle=\color[rgb]{0.3,0.4,0.3}\ttfamily,
  columns=fixed,
  numberstyle=\sffamily\scriptsize,
  backgroundcolor=\color[rgb]{0.95,0.95,0.95},
  frame=lines,
  framexleftmargin=5pt,
  numbers = left,
  numberstyle = \footnotesize
}
% ... until here -------------------------------------------

\newtheorem{theorem}{Theorem}

\begin{document}

% replace X and XXX with the number and title of the assignment:
\title{Exercise Sheet 10 - Task 5}
% DO NOT ADD YOUR NAME, only your student numbers:
\author{Simon Glomb 396387}
\date{09.07.2024}

\maketitle


%%%%%%%%%%%%%%%%%%%%%%%%%%%%%%%%%%%%%%%%%%%%%%%%%%%%%%%%%%%%%%%%%%%
% NOTE: You MUST read and follow Appendix E of the lecture notes! %
%%%%%%%%%%%%%%%%%%%%%%%%%%%%%%%%%%%%%%%%%%%%%%%%%%%%%%%%%%%%%%%%%%%
$L$ is the set of locations, $S$ the set of segments, $A$ the set of cars and $T$ the set of times. 
$L$ is the set of locations and $l\in L$ a location. $e\in S$ is a segment of a path in our instance, where $S$ is the set of segments. For our time horizon $T$ is $t\in T$ a time. We count the time in days. The cars are written as $i\in A$, where $A$ is the set of all cars. 
The binary variables $x_{iet}$ and $y_{ilt}$ tell us, on which segment $e$ or location $l$ a car $i$ is at time $t$. $K_{et}\in\mathbb{N}$ is the capacity of a segment $e$ at a time $t$, which is given in the instances as planned transports and vehicle capacity. The origin ${l_i}'\in L$ and destination ${l_i}^*\in L$ with available time $a_i\in T$ and due date $d_i\in T$ for each car $i$ are also given in the instances. Furthermore we use the binary variable $\bar d_i$, which takes the value 1 if a car has a due date and 0 if not. The binary variables $s_i$ and $w_i$ tell us for each car $i$, if the car is late and if the car arrives at its destination after at least $z_i\in \mathbb{N}$ days, that is double its network transportation time. We need those variables to compute the penalties given in the info sheet.  \\
In the formulation of our IP we would like to have as few variables as possibly, but for the sake of readability we introduce four variables as short terms for certain expressions. We call
\begin{equation}
   v_i = \bar d_i(b_i - d_i)
\end{equation}
the lateness of car $i$ with transport time 
\begin{equation}
   b_i = \sum_{t\in T}(\sum_{e\in S} x_{iet} + \sum_{l\in L}(y_{ilt} - y_{il_i^*t})).
\end{equation}

At last, we write the number of segments a car is at at a time $t$ as $X_{it} = \sum_{e\in S} x_{iet}$ and the number of locations a car is at as $Y_{it} = \sum_{l\in L} y_{ilt}$. We need this numbers to make sure, that a car is only on one transport or at one location at a time.\\

Here is our IP with the additional variables $v_i$, $b_i$, $X_{it}$ and $Y_{it}$:
--------------------------------------------------------------------------------
LP
--------------------------------------------------------------------------------
\begin{mini!}
{}{\sum_{i\in A} (100\bar d_i s_i + 25 s_i v_i + 5 w_i (b_i - z_i) + (1 - \bar d_i)b_i)\label{0}}
{}{}
\addConstraint{\sum_{i\in A} x_{iet}}{\leq  K_{et} \quad && \forall i\in A, e\in S, t\in T \label{1}}
\addConstraint{\sum_{t\in T} x_{iet}}{\leq LTH_e \quad && \forall i\in A, e\in S\label{2}}
\addConstraint{x_{iet}LTH_e}{\leq \sum_{t'\in T} x_{iet'} \quad && \forall e\in S, i\in A, t\in T\label{3}}
\addConstraint{\sum_{e\in \delta^-(l)}  x_{iet} + y_{ilt}}{= y_{ilt+1} + \sum_{e\in \delta^+(l)} x_{iet+1} \quad && \forall l\in L, i\in A, t\in T\setminus\{|T|\}\label{fb}}
\addConstraint{X_{it} + Y_{it}}{=1 \quad && \forall i\in A, t\in T\label{4}}
\addConstraint{y_{il_i't}}{= 1 \quad && \forall i\in A, t\in \{1,...,a_i\}\label{5}}
\addConstraint{v_i}{\leq Ms_i \quad && \forall i\in A\label{6}}
\addConstraint{s_i}{\leq  v_i + |v_i| \quad && \forall i\in A \label{7}}
\addConstraint{b_i - z_i}{\leq Mw_i \quad && \forall i\in A\label{8}}
\addConstraint{w_i}{\leq  b_i - z_i + |b_i - z_i| \quad  && \forall i\in A\label{9}}
\addConstraint{x_{iet}, y_{ilt}, s_i, w_i}{\in \{0,1\} \quad && \forall i\in A, e\in S, l\in L, t\in T,}
\end{mini!}

where $LTH_e$ is the duration of the transport (lead time hours).\\
------------------------------------------------------------------------------
Explanations
------------------------------------------------------------------------------
In our objective \ref{0} we sum over the costs of all cars, consisting of all penalties for each car $i$. The penalties contain a one time penalty for lateness (100euro), a penalty for each late day (25euro) and an additional penalty (5euro), if the car arrives at its destination more than double its network transportation time $z_i$. The penalties for lateness only make sense if car $i$ has a due date. If not, the costs for $i$ increase by 1(euro) per day, until the car is delivered. \\
Inequality \ref{1} describes the need to maintain the capacity on each segment. The inequalities \ref{2} and \ref{3} describe the fact that a car must stay on a segment until the transport duration is reached, so if a car $i$ is on a segment $e$ at a time $t$, it has to be there for $LTH_e$-long.
In equation \ref{fb} we establish the flow condition including cars staying at a location. 
Line \ref{4} is needed, because a car can only be in one place at a time. The equality \ref{5} guarantees that a car stays at its origin, until its available date is reached. Lines \ref{6} and \ref{7} guarantee that $s_i = 1$ if, and only if, the car $i$ is late. Analogously the inequalities \ref{8} and \ref{9} describe $w_i$. At last, all our variables are binary.\\
We use two big-M-constraints, where $M$ should be a great number with respect to time, for example $M = 2|T|$. 
\end{document}